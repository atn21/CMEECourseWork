\documentclass[12pt]{article}
\usepackage{graphicx}
\usepackage{float}
\usepackage{tabularx}

\newenvironment{conditions*}
  {\par\vspace{\abovedisplayskip}\noindent
   \tabularx{\columnwidth}{>{$}l<{$} @{${}={}$} >{\raggedright\arraybackslash}X}}
  {\endtabularx\par\vspace{\belowdisplayskip}}
  
\title{Is Florida getting warmer?}
\author{An Nguyen}
\date{\today}

\begin{document}

\maketitle

\section{Introduction}
This report investigates whether temperature significantly correlates with year, across years in Key West, Florida. 

The null hypothesis states that there is no strong correlation. The alternative hypothesis claims there is strong correlation, temperature does significantly correlates with year. A permutation analysis is used to calculate the p-value. A distribution of random correlation coefficients is generated. After sampling, to obtain the p-value, the number of test statistics which is as or more extreme than the initial test statistic is divided by the total number of test statistics calculated:

\begin{equation}
\frac{B+1}{M+1}
\end{equation}

where:
\begin{conditions*}
B    &  the number of random permutations in which a statistic greater or equal          than the observed one is obtained \\
M    &  the total number of random permutations sampled
\end{conditions*}

\section{Results}
The initial correlation coefficent was calcalutated to be 0.533. After some sufficient number of permutations, the approximate test statistic distribution was created, as shown below: 

\begin{figure}[H]
\includegraphics[keepaspectratio, width = 4in, height = 3in]{../results/Floridaplot.PDF}
\centering
\caption{Distribution of correlation coefficients for temperature and year, showing approximate p-value.}
\end{figure}

This distribution approximates all possible test statistic values we could have seen under the null hypothesis. No shuffled difference exceeds observed difference, i.e. p-value equals 0. This could be due to the p-value being too small, in such case Phipson and Smyth \cite{PhipsonSmyth+2010} recommends using $\frac{B+1}{M+1}$ to estimate p-value. Using this formula, p-value equals 0.00005002251 which is a lot lower than the normal threshold of 0.05. The evidence is not strong but it is possible that temperature does significantly correlate with year. 

\bibliographystyle{plain}
  
\bibliography{Floridabibli}

\end{document}
